\documentclass[a4paper]{article}
\PassOptionsToPackage{quiet}{xeCJK}
\usepackage[margin=1in]{geometry}
\usepackage{ctex}
\usepackage{xeCJK}
\usepackage{lipsum}
\usepackage{setspace}
\usepackage[noend,ruled,noline]{algorithm2e}
\title{\heiti\zihao{2} 算法设计与分析}
\author{\songti CS2008班   U202015533  徐瑞达}
\date{2022.03.12}
\begin{document}
\maketitle
\paragraph{定理描述:}
\subparagraph{对任意两个函数$f(n)$和$g(n)$,我们有$f(n)=\Theta(g(n))$,当且仅当$f(n)=O(g(n))$且$f(n)=\Omega(g(n))$.}
\paragraph{定理证明:}
\begin{flushleft}
    由$f(n)=\Theta(g(n))$可得:
    \centerline{当$n \ge n_{0}$时,有$0 \leq c_{1}g(n) \leq f(n) \leq c_{2}g(n)$}
    由$f(n)=O(g(n))$且$f(n)=\Omega(g(n))$可得:
    \centerline{对于任意的$n \geq n_{1}$,都有$0 \leq c_{3}g(n) \leq f(n)$}
    \centerline{而且对于任意的$n \geq n_{2}$,都有$0 \leq  f(n) \leq c_{4}g(n)$}
    如果$n_{3}=max(n_{1},n_{2})$并化简不等式,得到:
    \centerline{对于任意的$n \geq n_{3}$,都有$0 \leq c_{3}g(n) \leq f(n) \leq c_{4}g(n)$}
    这正是$\Theta$符号的定义,得证
\end{flushleft}
\end{document}